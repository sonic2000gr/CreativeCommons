%
% Creative commons license
%
\begin{center}
Copyright \copyright YYYY Όνομα Συγγραφέα\\
Το Παρόν Έργο παρέχεται υπό τους όρους της Άδειας:\\
\includegraphics[scale=0.2]{license/images/cc-logo}\\
\textbf{Αναφορά Δημιουργού-Μη Εμπορική Χρήση-Παρόμοια Διανομή 4.0 Διεθνής}\\
Το πλήρες κείμενο αυτής της άδειας είναι διαθέσιμο εδώ:\\
\url{http://creativecommons.org/licenses/by-nc-sa/4.0/}
\end{center}
\subsection*{Είστε ελεύθερος να:}
\noindent
\textbf{Διαμοιραστείτε} -- να αντιγράψετε και αναδιανείμετε το υλικό με οποιοδήπότε μέσο και μορφή.\\
\textbf{Προσαρμόσετε} -- να αναμείξετε, μετασχηματίσετε και να επεκτείνετε το υλικό.\\

Ο αδειοδότης δεν μπορεί να σας αφαιρέσει αυτές τις ελευθερίες όσο ακολουθείτε τους όρους της παρούσας άδειας.
\subsection*{Ύπο τους ακόλουθους όρους:}

\vspace{1em}
\noindent
\parbox{1.5cm}{\includegraphics[scale=1.5]{license/images/cc_by_30}}
\parbox{10.5cm}{\textbf{Αναφορά Δημιουργού} -- Θα πρέπει να αναφέρετε \textbf{τον δημιουργό του έργου}, να παρέχετε σύνδεσμο προς αυτή την άδεια, και να \textbf{υποδείξετε τυχόν αλλαγές}. Μπορείτε να το κάνετε με οποιοδήποτε εύλογο μέσο, αλλά όχι με τρόπο που να υπονοεί ότι ο αδειοδότης επικροτεί εσάς ή τη χρήση του έργου από εσάς.}

\vspace{1em}
\noindent
\parbox{1.5cm}{\includegraphics[scale=1.5]{license/images/cc_nc_30}}
\parbox{10.5cm}{\textbf{Μη Εμπορική Χρήση} --  Δεν μπορείτε να χρησιμοποιήσετε το υλικό για \textbf{εμπορικούς σκοπούς}.}

\vspace{1em}
\noindent
\parbox{1.5cm}{\includegraphics[scale=1.5]{license/images/cc_sa_30}}
\parbox{10.5cm}{\textbf{Παρόμοια Διανομή}  -- Αν αναμείξετε, μετασχηματίσετε ή επεκτείνετε το υλικό, θα πρέπει να διανείμετε τις αλλαγές σας υπό την \textbf{ίδια άδεια} με το πρωτότυπο έργο.}

\vspace{1em}
\noindent
\parbox{1.5cm}{\ }
\parbox{10.5cm}{\textbf{Όχι επιπλέον περιορισμοί} -- Δεν μπορείτε να εφαρμόσετε νομικούς όρους ή \textbf{τεχνικά μέσα} που να περιορίζουν νομικά τους άλλους να πράξουν σύμφωνα με τις ελευθερίες αυτής της άδειας.}
\subsection*{Σημειώσεις:}
\noindent
Δεν χρειάζεται να ακολουθήσετε την άδεια για τμήματα του υλικού που θεωρούνται δημόσια γνώση (public domain) ή όπου η χρήση τους επιτρέπεται εξαιτίας μιας \textbf{εξαίρεσης ή περιορισμού}.\\

\noindent
Δεν δίνονται εγγυήσεις. Η άδεια ίσως να μη σας δίνει όλα τα δικαιώματα για την επιδιωκόμενη χρήση. Για παράδειγμα, επιπλέον δικαιώματα όπως \textbf{δημοσιότητα, ιδιωτικότητα, ή ηθικά δικαιώματα} μπορεί να επιβάλλουν περιορισμούς στη χρήση του υλικού.\\
\line(1,0){390}\\\\
\noindent
Το παρόν έργο στοιχειοθετήθηκε σε \XeLaTeX. Ο πηγαίος κώδικας του είναι διαθέσιμος στην παρακάτω τοποθεσία:
\begin{center}
\url{http://the.source.code/repo}
\end{center}
\newpage
